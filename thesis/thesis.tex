\documentclass{uvamscse}
\usepackage{color}

\usepackage{listings}

\lstdefinelanguage{sdf}{%
  numbers=none,
  morekeywords={module,imports,exports,sorts,context,lexical,free,syntax,==,=,+,-,left,cons,prefer,avoid,bracket},
  columns=flexible,
  morestring=[b]",
  basicstyle=\footnotesize\mdseries,
  literate={->}{{\,\,$\to$\,\,}}1
}

\lstdefinelanguage{dcg}{%
  numbers=none,
  morekeywords={},
  morestring=[b]",
  basicstyle=\footnotesize\mdseries,
  columns=flexible,
}

\lstdefinelanguage{prolog},
  literate={:-}{{\,$\Leftarrow$\,\,}}1 {-->}{{$\to$\,}}1
}

\lstdefinestyle{mono}{
  basicstyle=\footnotesize\ttfamily
}

\lstdefinestyle{monosmall}{
  basicstyle=\footnotesize\ttfamily
}

\lstdefinelanguage{javascript}{
  keywords={typeof, new, true, false, catch, function, return, null, catch, switch, var, if, in, while, do, else, case, break},
  keywordstyle=\color{blue}\bfseries,
  ndkeywords={class, export, boolean, throw, implements, import, this},
  ndkeywordstyle=\color{darkgray}\bfseries,
  identifierstyle=\color{black},
  sensitive=false,
  comment=[l]{//},
  morecomment=[s]{/*}{*/},
  commentstyle=\color{purple}\ttfamily,
  stringstyle=\color{red}\ttfamily,
  morestring=[b]',
  morestring=[b]"
}

\lstdefinelanguage{json}{
    basicstyle=\normalfont\ttfamily,
    numbers=left,
    numberstyle=\scriptsize,
    stepnumber=1,
    numbersep=8pt,
    showstringspaces=false,
    breaklines=false,
    % frame=lines,
    backgroundcolor=\color{white},
    literate=
     *{0}{{{\color{black}0}}}{1}
      {1}{{{\color{black}1}}}{1}
      {2}{{{\color{black}2}}}{1}
      {3}{{{\color{black}3}}}{1}
      {4}{{{\color{black}4}}}{1}
      {5}{{{\color{black}5}}}{1}
      {6}{{{\color{black}6}}}{1}
      {7}{{{\color{black}7}}}{1}
      {8}{{{\color{black}8}}}{1}
      {9}{{{\color{black}9}}}{1}
      {:}{{{\color{black}{:}}}}{1}
      {,}{{{\color{black}{,}}}}{1}
      {\{}{{{\color{black}{\{}}}}{1}
      {\}}{{{\color{black}{\}}}}}{1}
      {[}{{{\color{black}{[}}}}{1}
      {]}{{{\color{black}{]}}}}{1},
}

\lstset{%
  frame=none,
  xleftmargin=2pt,
  stepnumber=1,
  numbers=left,
  numbersep=7pt,
  numberstyle=\ttfamily\scriptsize\color[gray]{0.3},
  belowcaptionskip=\bigskipamount,
  captionpos=b,
%  escapeinside={*'}{'*},
  % language=fl,
  tabsize=2,
  emphstyle={\bf},
  stringstyle=\itshape,
  showspaces=false,
  keywordstyle=\bfseries\rmfamily,
  columns=flexible,
  basicstyle=\small\mdseries,
  showstringspaces=false,
}

\newcommand{\cmd}[1]{\texttt{$\backslash$#1}}

\title{Geographically-aware scaling for real-time persistent websocket applications.}
% \coverpic[100pt]{figures/terminal.png}
\subtitle{Master's Project in Software Engineering}
\date{Summer 2015}

\author{Lukasz Harezlak}
\authemail{lukasz.harezlak@gmail.com}
\host{Instamrkt, \url{https://instamrkt.com}}

\abstract{
  This section summarises the content of the thesis for potential readers who do not have time to read it whole,
  or for those undecided whether to read it at all. Sum up the following aspects:

  \begin{itemize}
    \item relevance and motivation for the research
    \item research question(s) and a brief description of the research method
    \item results, contributions and conclusions
  \end{itemize}

Kent Beck~\cite{JohnsonBBCGW93} proposes to have four sentences in a good abstract:

  \begin{enumerate}
    \item The first states the problem.
    \item The second states why the problem is a problem.
    \item The third is the startling sentence.
    \item The fourth states the implication of the startling sentence.
  \end{enumerate}
}

\begin{document}
\maketitle

%%%%%%%%%%%%%%%%%%%%%%%%%%%%%%%%%%%%%%%%%%%%%%%%%%%%%%%%%%%%%%%%%%%%%%%%%%%%%%%%
% Let the juicy stuff start.
%%%%%%%%%%%%%%%%%%%%%%%%%%%%%%%%%%%%%%%%%%%%%%%%%%%%%%%%%%%%%%%%%%%%%%%%%%%%%%%%

%%%%%%%%%%%%%%%%%%%%%%%%%%%%%%%%%%%%%%%%%%%%%%%%%%%%%%%%%%%%%%%%%%%%%%%%%%%%%%%%
\chapter{Introduction}
Here we write about scalability.
Importance of measuring in the clouds and testing optimal architectures.

\section{Initial Study}
What we found out researching.

\section{Problem Statement}
State the general problem that you are trying to solve with this research.
\subsection{Research Questions}
Precise research questions here.
\subsection{Solution Outline}
How our solution works in short.
\subsection{Research Method}
Why we went for a controlled experiment.

\section{Contributions}
Case study on AWS.
Architecture decomposition analysis.
Deliverable piece of code - metrics framework and auto-scaling scripts.

\section{Related Work}
The big paper on measuring scalability in the clouds.
The big paper on deploying different scalability structures.
Small mentions of the others.

\section{Outline}
Here we outline the structure of the thesis. A short paragraph on what happens in each chapter.
%%%%%%%%%%%%%%%%%%%%%%%%%%%%%%%%%%%%%%%%%%%%%%%%%%%%%%%%%%%%%%%%%%%%%%%%%%%%%%%%

%%%%%%%%%%%%%%%%%%%%%%%%%%%%%%%%%%%%%%%%%%%%%%%%%%%%%%%%%%%%%%%%%%%%%%%%%%%%%%%%
\chapter{Background}
General stuffs go here. Look through Scalability Research in drive for tons of information.

\section{Scalability}
General.

\subsection{Measuring Scalability} \label{Measuring Scalability}
What we have from literature study.

\subsection{Data-layer Scalability}

\subsection{Websocket Scalability}

\subsection{Geographical Distribution}
%%%%%%%%%%%%%%%%%%%%%%%%%%%%%%%%%%%%%%%%%%%%%%%%%%%%%%%%%%%%%%%%%%%%%%%%%%%%%%%%

%%%%%%%%%%%%%%%%%%%%%%%%%%%%%%%%%%%%%%%%%%%%%%%%%%%%%%%%%%%%%%%%%%%%%%%%%%%%%%%%
\chapter{Experiment outline}

\begin{enumerate}
  \item Baseline architecture on a local network.
  \item Baseline architecture deployed in the cloud.
  \item Improved architecture in the cloud.
\end{enumerate}

All of them measured with the same set of metrics with a prepared framework for gathering them.

Details below.

\section{Baseline architecture on a local network}
Architecture diagrams.
Technical details - local server capabilities.
Load testing.

\section{Baseline architecture deployed in the cloud}

\section{Improved architecture deployed in the cloud}

%%%%%%%%%%%%%%%%%%%%%%%%%%%%%%%%%%%%%%%%%%%%%%%%%%%%%%%%%%%%%%%%%%%%%%%%%%%%%%%%

%%%%%%%%%%%%%%%%%%%%%%%%%%%%%%%%%%%%%%%%%%%%%%%%%%%%%%%%%%%%%%%%%%%%%%%%%%%%%%%%
\chapter{Scalability Measurements}
The previous description \ref{Measuring Scalability} concerned general stuff.
Here we describe the metrics we settled for.

\section{Selected model}
[3] With CRUMs, SPMs etc.

\section{Baseline, Local network}
Lab notes from 11 of May go here:

Basic metrics
\begin{itemize}
  \item cpu, memory, disk usage (pidstat / CloudWatch)
  \item network i/o (wireshark, list others analyzed)
  \item latency, throughput, concurrent connections, messages dropped?
\end{itemize}

\subsection{Load Testing}
all the analyzed tools, list also in lab notes from may 11

\section{Baseline, Deployed in the cloud}

\section{Improved, Deployed in the cloud}
%%%%%%%%%%%%%%%%%%%%%%%%%%%%%%%%%%%%%%%%%%%%%%%%%%%%%%%%%%%%%%%%%%%%%%%%%%%%%%%%

%%%%%%%%%%%%%%%%%%%%%%%%%%%%%%%%%%%%%%%%%%%%%%%%%%%%%%%%%%%%%%%%%%%%%%%%%%%%%%%%
\chapter{Experiment results}
%%%%%%%%%%%%%%%%%%%%%%%%%%%%%%%%%%%%%%%%%%%%%%%%%%%%%%%%%%%%%%%%%%%%%%%%%%%%%%%%

%%%%%%%%%%%%%%%%%%%%%%%%%%%%%%%%%%%%%%%%%%%%%%%%%%%%%%%%%%%%%%%%%%%%%%%%%%%%%%%%
\chapter{Evaluation}
%%%%%%%%%%%%%%%%%%%%%%%%%%%%%%%%%%%%%%%%%%%%%%%%%%%%%%%%%%%%%%%%%%%%%%%%%%%%%%%%

%%%%%%%%%%%%%%%%%%%%%%%%%%%%%%%%%%%%%%%%%%%%%%%%%%%%%%%%%%%%%%%%%%%%%%%%%%%%%%%%
\chapter{Conclusion}
%%%%%%%%%%%%%%%%%%%%%%%%%%%%%%%%%%%%%%%%%%%%%%%%%%%%%%%%%%%%%%%%%%%%%%%%%%%%%%%%

%%%%%%%%%%%%%%%%%%%%%%%%%%%%%%%%%%%%%%%%%%%%%%%%%%%%%%%%%%%%%%%%%%%%%%%%%%%%%%%%
\chapter{Further work}
%%%%%%%%%%%%%%%%%%%%%%%%%%%%%%%%%%%%%%%%%%%%%%%%%%%%%%%%%%%%%%%%%%%%%%%%%%%%%%%%

%%%%%%%%%%%%%%%%%%%%%%%%%%%%%%%%%%%%%%%%%%%%%%%%%%%%%%%%%%%%%%%%%%%%%%%%%%%%%%%%
% BELOW WE HAVE ORIGINAL TEMPLATE STUFF THAT NEEDS TO BE REMOVED
%%%%%%%%%%%%%%%%%%%%%%%%%%%%%%%%%%%%%%%%%%%%%%%%%%%%%%%%%%%%%%%%%%%%%%%%%%%%%%%%
\chapter{Everything below that needs to be removed (except for bib)}
\chapter{Front Matter}

The first thing is to connect the class by saying:

\begin{snippet}
\begin{verbatim}
\documentclass{uvamscse}
\end{verbatim}
\end{snippet}

\section{Title}

Specify the title of the thesis with \cmd{title} and \cmd{subtitle} commands:

\begin{snippet}
\begin{verbatim}
\title{MetaThesis}
\subtitle{A Thesis Template Leading by Example}
\end{verbatim}
\end{snippet}

Any thesis can survive without a \cmd{subtitle}, but the \cmd{title} is mandatory.

\section{Author}

Introduce yourself with \cmd{author} and \cmd{authemail}:

\begin{snippet}
\begin{verbatim}
\author{Vadim Zaytsev}
\authemail{vadim@grammarware.net}
\end{verbatim}
\end{snippet}

Again, \cmd{authemail} is not mandatory. If you need anything fancier, just put it inside \cmd{author}.

\begin{snippet}
\begin{verbatim}
\author{Vadim Zaytsev\footnote{Yes, that one.}}
\end{verbatim}
\end{snippet}

The footnote would be printed on the bottom of the title page, and will be
referred to by a symbol, not by a number as any footnotes within the main
document body.

\section{Date}

By default, the date inserted in your PDF is the day of the build, e.g., ``March 25, 2014''. If you want it to be formatted differently or be more vague or outright fake, use \cmd{date}:

\begin{snippet}
\begin{verbatim}
\date{Spring 2014}
\end{verbatim}
\end{snippet}

The argument is just a string, the format is unrestricted:

\begin{snippet}
\begin{verbatim}
\date{Tomorrow. Honestly.}
\end{verbatim}
\end{snippet}

\section{Host}

If your hosting organisation is not the UvA, specify it with \cmd{host}. The
logo on the bottom of the title page will still be the UvA one, because this
is the organisation guaranteeing your degree.

\begin{snippet}
\begin{verbatim}
\host{Grammarware, Inc., \url{http://grammarware.github.io}}
\end{verbatim}
\end{snippet}

NB: footnotes will not work, unless you know how to \cmd{protect} them.

\section{Cover picture}

If the first page of your thesis looks too blunt, add a picture to it:

\begin{snippet}
\begin{verbatim}
\coverpic{figures/terminal.png}
\end{verbatim}
\end{snippet}

You can even specify the picture's width as an optional argument:

\begin{snippet}
\begin{verbatim}
\coverpic[100pt]{figures/terminal.png}
\end{verbatim}
\end{snippet}

How these three options look, you can see from \autoref{fig:titles}.

\begin{figure}[t]
  \fbox{\includegraphics[width=.25\textwidth]{figures/title1.pdf}}
  \hfill
  \fbox{\includegraphics[width=.25\textwidth]{figures/title2.pdf}}
  \hfill
  \fbox{\includegraphics[width=.25\textwidth]{figures/title3.pdf}}
  \caption{A hypothetical thesis title page without a cover picture (on the left), with an overly large one (in the centre) and with a tiny pic (on the right).}
  \label{fig:titles}
\end{figure}

\section{Abstract}

A thesis is fine without an abstract, if you do not feel like writing it and
your supervisor does not feel like enforcing it. If you do want an abstract,
make it with the \cmd{abstract} command:

\begin{snippet}
\begin{verbatim}
\abstract{This is not a thesis.}
\end{verbatim}
\end{snippet}

The abstract is just like any other section of your thesis, so you can use any
\LaTeX\ tricks there. If you think that the name ``abstract'' is too abstract
for your abstract, you can still use \cmd{abstract} without being too
abstract:

\begin{snippet}
\begin{verbatim}
\abstract[Confession]{I am a cenosillicaphobiac.}
\end{verbatim}
\end{snippet}

Kent Beck~\cite{JohnsonBBCGW93} proposes to have four sentences in a good abstract:

\begin{enumerate}
  \item The first states the problem.
  \item The second states why the problem is a problem.
  \item The third is the startling sentence.
  \item The fourth states the implication of the startling sentence.
\end{enumerate}

In practice, each of these ``sentences'' can be longer than an actual
sentence, but it is in general a good rule of thumb to condense the summary of
your thesis into these four tiny messages. Do not write too much, make it
tweetable.

%%%%%%%%%%%%%%%%%%%%%%%%%%%%%%%%%%%%%%%%%%%%%%%%%%%%%%%%%%%%%%%%%%%%%%%%%%%%%%%
\chapter{Core Chapters}

The structure of your thesis is up to you and your supervisor. Whatever you
do, do not consider the guidelines below as dogmas.

\section{Classic structure}

\begin{description}
  \item[Problem statement and motivation.]
  You describe in detail what problem the research is addressing, and what is
the motivation to address this problem. There is a concise and objective
statement of the research questions, hypotheses and goals. It is made clear
why these questions and goals are important and relevant to the world outside
the university (assuming it exists). You can already split the main research
question into subquestions in this chapter. This section also describes an
analysis of the problem: where does it occur and how, how often, and what are
the consequences? An important part is also to scope the research: what
aspects are included and what aspects are deliberately left out, and why?
  \item[Research method.]
  Here you describe the methods used to answer the research questions. A good
structure of this section often follows the subquestions by providing a method
for each. The research method needs a thorough motivation grounded in theory
in order to be acceptable. As a part of the method, you can introduce a number
of hypotheses --- these will be tested by the research, using the methods
described here. An important part of this section is validation. How will you
evaluate and validate the outcomes of the research?
  \item[Background and context.]
  This chapter contains all the information needed to put the thesis into
context. It is common to use a revised version of your literature survey for
this purpose. It is important to explicitly refer from your text to sources
you have used, they will be listed in your bibliography. For example, you can
write ``A small number of programming languages account for most language
use~\cite{MeyerovichR2013}'', where the following entry would be included in
your bibliography:
\begin{quote}
\cite{MeyerovichR2013} Leo A. Meyerovich and Ariel S. Rabkin. Empirical Analysis of Programming Language Adoption. In \emph{Proceedings of the 2013 ACM SIGPLAN International Conference on Object Oriented Programming Systems Languages and Applications}, OOPSLA, pages 1--18. ACM, 2013. \doi{10.1145/2509136.2509515}.
\end{quote}
Have a look at \autoref{sec:biblio} to learn more about citation.
  \item[Research.]
  This chapter reports on the execution of the research method as described in
an earlier chapter. If the research has been divided into phases, they are
introduced, reported on and concluded individually. If needed, this chapter
could be split up to balance out the sizes of all chapters.
  \item[Results.]
  This chapter presents and clarifies the results obtained during the
  research. The focus should be on the factual results, not the interpretation
  or discussion. Tables and graphics should be used to increase the clarity of
  the results where applicable.
  \item[Analysis and conclusions.]
  This chapter contains the analysis and interpretation of the results. The
  research questions are answered as best as possible with the results that
  were obtained. The analysis also discussed parts of the questions that were
  left unanswered.

  An important topic is the validity of the results. What methods of
  validation were used? Could the results be generalised to other cases? What
  threats to validity can be identified? There is room here to discuss the
  results of related scientific literature here as well. How do the results
  obtained here relate to other work, and what consequences are there? Did
  your approach work better or worse? Did you learn anything new compared to
  the already existing body of knowledge? Finally, what could you say in
  hindsight on the research approach by followed? What could have done better?
  What lessons have been learned? What could other researchers use from your
  experience? A separate section should be devoted to ``future work'', i.e.,
  possible extension points of your work that you have identified. Even other
  researchers should be able to use those as a starting point.
\end{description}

\section{Reporting on replications}

Here are the guidelines to report on replicated studies~\cite{Carver10}:

\begin{description}
  \item[Information about the original study]~\\
    \begin{description}
    \item[Research question(s)] that were the basis for the design
    \item[Participants,] their number and any other relevant characteristics
    \item[Design] as a graphical or textual description of the experimental design
    \item[Artefacts,] the description of them and/or links to the artefacts used
    \item[Context variables] as any important details that affected the design of the study or interpretation of the
results
    \item[Summary of the results] in a brief overview of the major findings
    \end{description}
  %
  \item[Information about the replication]~\\
    \begin{description}
    \item[Motivation for conducting the replication] as a
description of why the replication was conducted:
to validate the results, to broaden the results by
changing the participant pool or the artifacts.
    \item[Level of interaction with original experimenters.]
The level of interaction between the original experimenters and the
replicators should be reported. This interaction could range from none (i.e.
simply read the  paper) to them being the same people. There is quite a lot of
discussion of the level of interaction allowed for the replication to be
``successful'', but this level should be reported even without  addressing
the controversy.
    \item[Changes to the original experiment.] Any changes made to the
design, participants, artifacts, procedures, data collected and/or analysis
techniques should be  discussed along with the motivation for the change.
    \end{description}
  \item[Comparison of results to original]~\\
    \begin{description}
    \item[Consistent results,] when replication results supported
results from the original study, and
    \item[Differences in results,] when results from the replication
did not coincide with the results from the original study.
Authors should also discuss how changes made to the
experimental design (see above) may have caused
these differences.
    \end{description}
    \item[Drawing conclusions across studies]
\end{description}

NB: this section contains portions of text repeated directly from Carver~\cite{Carver10} and
only slightly massaged. Do not do this for your thesis, write your own thoughts down.

\section{\LaTeX\ details}

\subsection{Environments}

A \LaTeX\ environment is something with opening and closing tags, which look
like \cmd{begin}\{\texttt{name}\} and \cmd{end}\{\texttt{name}\}. Some useful
environments to know:

\begin{center}
\begin{tabular}{ll}
  \texttt{itemize}      & bullet lists\\
  \texttt{enumerate}    & numbered lists\\
  \texttt{description}  & definition lists\\
  \hline
  \texttt{center}       & centered line elements\\
  \texttt{flushright}   & right aligned lines\\
  \texttt{flushleft}    & left aligned lines\\
  \hline
  \texttt{tabular}      & table\\
  \texttt{longtable}    & multi-page table (needs the \texttt{longtable} package)\\
  \texttt{sideways}     & rotates some text\\
  \texttt{quote}        & block quote\\
  \texttt{verbatim}     & unformatted text\\
  \texttt{minipage}     & compound box with elements inside\\
  \texttt{boxedminipage}& compound box with elements inside and a border around it\\
  \hline
  \texttt{table}        & floating table (needs to have \texttt{tabular} nested inside)\\
  \texttt{figure}       & floating figure\\
  \texttt{sourcecode}   & floating listing\\
  \hline
  \texttt{equation}     & mathematical equation\\
  \texttt{lstlisting}   & pretty-printed syntax highligted listing\\
  \texttt{multline}     & mathematical equation spanning over multiple lines\\
  \texttt{eqnarray}     & system of mathematical equations\\
  \texttt{gather}       & bundled mathematical equations\\
  \texttt{align}        & bundled and aligned mathematical equations\\
  \texttt{array}        & matrix\\
  \texttt{CD}           & commutative diagrams\\
\end{tabular}
\end{center}

\section{Listings}

\begin{sourcecode}
\begin{lstlisting}[language=prolog]
define(Ps1,G1,G2)
 :-
    usedNs(G1,Uses),
    ps2n(Ps1,N),
    require(
      member(N,Uses),
      'Nonterminal ~q must not be fresh.',
      [N]),
    new(Ps1,N,G1,G2),
    !.
\end{lstlisting}
\caption{Code in Prolog}
\end{sourcecode}

\begin{sourcecode}
\begin{lstlisting}[language=sdf]
module Syntax

imports Numbers
imports basic/Whitespace

exports
  sorts
    Program Function Expr Ops Name Newline

  context-free syntax
    Function+                          -> Program
    Name Name+ "=" Expr Newline+       -> Function
    Expr Ops Expr                      -> Expr      {left,prefer,cons(binary)}
    Name Expr+                         -> Expr      {avoid,cons(apply)}
    "if" Expr "then" Expr "else" Expr  -> Expr      {cons(ifThenElse)}
    "(" Expr ")"                       -> Expr      {bracket}
    Name                               -> Expr      {cons(argument)}
    Int                                -> Expr      {cons(literal)}
    "-"                                -> Ops       {cons(minus)}
    "+"                                -> Ops       {cons(plus)}
    "=="                               -> Ops       {cons(equal)}
\end{lstlisting}
\caption{Code in SDF}
\end{sourcecode}

\begin{sourcecode}
\begin{lstlisting}[language=Java]
import types.*;
import org.antlr.runtime.*;

public class TestEvaluator
    public static void main(String[] args) throws Exception {

        // Parse file to program
        ANTLRFileStream input = new ANTLRFileStream(args[0]);
        FLLexer lexer = new FLLexer(input);
        CommonTokenStream tokens = new CommonTokenStream(lexer);
        FLParser parser = new FLParser(tokens);
        Program program = parser.program();

        // Parse sample expression
        input = new ANTLRFileStream(args[1]);
        lexer = new FLLexer(input);
        tokens = new CommonTokenStream(lexer);
        parser = new FLParser(tokens);
        Expr expr = parser.expr();

        // Evaluate program
        Evaluator eval = new Evaluator(program);
        int expected = Integer.parseInt(args[2]);
\end{lstlisting}
\caption{Code in Java}
\end{sourcecode}

\begin{sourcecode}
\begin{lstlisting}[style=mono,language=Python]
#!/usr/local/bin/python
# wiki: BGF
import os
import sys
import slpsns
import elementtree.ElementTree as ET

# root::nonterminal* production*
class Grammar:
  def __init__(self):
    self.roots = []
    self.prods = []
  def parse(self,fname):
    self.roots = []
    self.prods = []
    self.xml = ET.parse(fname)
    for e in self.xml.findall('root'):
      self.roots.append(e.text)
    for e in self.xml.findall(slpsns.bgf_('production')):
      prod = Production()
      prod.parse(e)
      self.prods.append(prod)
\end{lstlisting}
\caption{Code in Python}
\end{sourcecode}

\chapter{Literature}\label{sec:biblio}

\textsc{Bib}TeX\ is a JSON-like format for bibliographic entries. Encode each
source once as a \textsc{Bib}\TeX\ entry, give it a name and refer to it from
any place in your thesis. The bibliography at the end of the thesis will be
compiled automatically from those entries that are referenced at least once,
it will also be automatically sorted and fancyfied (URLs, DOIs, etc).

DOI is a digital object identifier, it is uniquely and immutably assigned to
any paper published in a well-established journal or conference proceedings
and can be used to refer to it. When used in a browser, it resolves to a
publisher's website where paper can be obtained. Including DOIs in citations
is considered good practice and lets the readers of your thesis get to the
text of the paper in one click. Books do not have DOIs, only ISBNs; some
workshop proceedings and most unofficial publications do not have DOIs. If you
want to get a DOI assigned to your work such as a piece of code, upload it to
\href{http://www.figshare.com}{FigShare}.

Keys in key-value pairs within each \textsc{Bib}\TeX\ entry are never quoted,
values usually are, but can also be included within curly brackets or left as
is, which works fine for numbers (e.g., years). If you want to preserve the
value from any adjustments (e.g., no recapitalisation in titles), use curlies
\emph{and} quotes. Separate authors and editors by ``and'', which will
automatically be mapped to commas or left as ``and''s as necessary.

\section{Books}

\cite{GruneJacobs} is just as good as the Dragon Book, but newer and has an
awesome extended bibliography available for free.

\begin{snippet}
\begin{verbatim}
@book{GruneJacobs,
  author    = "D. Grune and C. J. H. Jacobs",
  title     = "{Parsing Techniques: A Practical Guide}",
  series    = "Monographs in Computer Science",
  edition   = 2,
  publisher = "Springer",
  url       = "http://www.cs.vu.nl/~dick/PT2Ed.html",
  year      = 2008,
}
\end{verbatim}
\end{snippet}

\section{Journal papers}

Not all TOSEM papers are hard to read~\cite{GrammarwareAgenda}.

\begin{snippet}
\begin{verbatim}
@article{GrammarwareAgenda,
  author      = "Paul Klint and Ralf L{\"a}mmel and Chris Verhoef",
  title       = "{Toward an Engineering Discipline for Grammarware}",
  journal     = "ACM Transactions on Software Engineering Methodology (TOSEM)",
  volume      = 14,
  number      = 3,
  year        = 2005,
  pages       = "331--380",
}
\end{verbatim}
\end{snippet}

\section{Conference papers}

There is no limit to how many grammars can be used in one paper, but the
current record stands at 569~\cite{Micropatterns2013}.

\begin{snippet}
\begin{verbatim}
@inproceedings{Micropatterns2013,
  author = "Vadim Zaytsev",
  title = "{Micropatterns in Grammars}",
  booktitle = "{Proceedings of the Sixth International Conference on Software Language Engineering
                (SLE 2013)}",
  year = 2013,
  editor = "Martin Erwig and Richard F. Paige and Eric Van Wyk",
  volume = "8225",
  series = "LNCS",
  pages = "117--136",
  address = "Switzerland",
  month = oct,
  publisher = "Springer International Publishing",
  doi = "10.1007/978-3-319-02654-1_7",
}
\end{verbatim}
\end{snippet}

\section{Theses}

The seventh PhD student of Paul Klint was Jan Rekers~\cite{Rekers92}.

\begin{snippet}
\begin{verbatim}
@phdthesis{Rekers92,
 author   = "J. Rekers",
 title    = "{Parser Generation for Interactive Environments}",
 school   = "University of Amsterdam",
 year     = 1992,
 url      = "http://homepages.cwi.nl/~paulk/dissertations/Rekers.pdf",
}
\end{verbatim}
\end{snippet}

There is also \texttt{mastersthesis} type with exactly the same structure for
referring to Master's theses.

\section{Technical reports}

The original seminal work introducing two-level grammars was never published
in any book or conference, but there is a technical report explaining
it~\cite{Wijngaarden65}. SMC, or \emph{Stichting Matematisch Centrum}, was the
old name of CWI fifty years ago.

\begin{snippet}
\begin{verbatim}
@techreport{Wijngaarden65,
        author      = "Adriaan van Wijngaarden",
        title       = "{Orthogonal Design and Description of a Formal Language}",
        month       = oct,
        year        = 1965,
        institution = "SMC",
        type        = "{MR 76}",
        url         = "http://www.fh-jena.de/~kleine/history/languages/VanWijngaarden-MR76.pdf",
}
\end{verbatim}
\end{snippet}

\section{Wikipedia}

You do not refer to Wikipedia from academic writing, it works the other way around.

\section{Anything else}

You can refer to pretty much anything (websites, blog posts, software) through
\texttt{misc} type of entry~\cite{ANTLR}:

\begin{snippet}
\begin{verbatim}
@misc{ANTLR,
 author       = "Terence Parr",
 title        = "{ANTLR---ANother Tool for Language Recognition}",
 howpublished = "Software",
 url          = "http://antlr.org",
 year         = "2008"
}
\end{verbatim}
\end{snippet}

{%\tiny
\bibliographystyle{alphaurl}
\bibliography{thesis}
}

\end{document}
