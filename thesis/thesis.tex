\documentclass{uvamscse}
\usepackage{color}

\usepackage{listings}

\lstdefinelanguage{sdf}{%
  numbers=none,
  morekeywords={module,imports,exports,sorts,context,lexical,free,syntax,==,=,+,-,left,cons,prefer,avoid,bracket},
  columns=flexible,
  morestring=[b]",
  basicstyle=\footnotesize\mdseries,
  literate={->}{{\,\,$\to$\,\,}}1
}

\lstdefinelanguage{dcg}{%
  numbers=none,
  morekeywords={},
  morestring=[b]",
  basicstyle=\footnotesize\mdseries,
  columns=flexible,
}

\lstdefinelanguage{prolog},
  literate={:-}{{\,$\Leftarrow$\,\,}}1 {-->}{{$\to$\,}}1
}

\lstdefinestyle{mono}{
  basicstyle=\footnotesize\ttfamily
}

\lstdefinestyle{monosmall}{
  basicstyle=\footnotesize\ttfamily
}

\lstdefinelanguage{javascript}{
  keywords={typeof, new, true, false, catch, function, return, null, catch, switch, var, if, in, while, do, else, case, break},
  keywordstyle=\color{blue}\bfseries,
  ndkeywords={class, export, boolean, throw, implements, import, this},
  ndkeywordstyle=\color{darkgray}\bfseries,
  identifierstyle=\color{black},
  sensitive=false,
  comment=[l]{//},
  morecomment=[s]{/*}{*/},
  commentstyle=\color{purple}\ttfamily,
  stringstyle=\color{red}\ttfamily,
  morestring=[b]',
  morestring=[b]"
}

\lstdefinelanguage{json}{
    basicstyle=\normalfont\ttfamily,
    numbers=left,
    numberstyle=\scriptsize,
    stepnumber=1,
    numbersep=8pt,
    showstringspaces=false,
    breaklines=false,
    % frame=lines,
    backgroundcolor=\color{white},
    literate=
     *{0}{{{\color{black}0}}}{1}
      {1}{{{\color{black}1}}}{1}
      {2}{{{\color{black}2}}}{1}
      {3}{{{\color{black}3}}}{1}
      {4}{{{\color{black}4}}}{1}
      {5}{{{\color{black}5}}}{1}
      {6}{{{\color{black}6}}}{1}
      {7}{{{\color{black}7}}}{1}
      {8}{{{\color{black}8}}}{1}
      {9}{{{\color{black}9}}}{1}
      {:}{{{\color{black}{:}}}}{1}
      {,}{{{\color{black}{,}}}}{1}
      {\{}{{{\color{black}{\{}}}}{1}
      {\}}{{{\color{black}{\}}}}}{1}
      {[}{{{\color{black}{[}}}}{1}
      {]}{{{\color{black}{]}}}}{1},
}

\lstset{%
  frame=none,
  xleftmargin=2pt,
  stepnumber=1,
  numbers=left,
  numbersep=7pt,
  numberstyle=\ttfamily\scriptsize\color[gray]{0.3},
  belowcaptionskip=\bigskipamount,
  captionpos=b,
%  escapeinside={*'}{'*},
  % language=fl,
  tabsize=2,
  emphstyle={\bf},
  stringstyle=\itshape,
  showspaces=false,
  keywordstyle=\bfseries\rmfamily,
  columns=flexible,
  basicstyle=\small\mdseries,
  showstringspaces=false,
}

\newcommand{\cmd}[1]{\texttt{$\backslash$#1}}

\title{Geographically-aware scaling for real-time persistent websocket applications.}
% \coverpic[100pt]{figures/terminal.png}
\subtitle{Master's Project in Software Engineering}
\date{Summer 2015}

\author{Lukasz Harezlak}
\authemail{lukasz.harezlak@gmail.com}
\host{Instamrkt, \url{https://instamrkt.com}}

\abstract{
  This section summarises the content of the thesis for potential readers who do not have time to read it whole,
  or for those undecided whether to read it at all. Sum up the following aspects:

  \begin{itemize}
    \item relevance and motivation for the research
    \item research question(s) and a brief description of the research method
    \item results, contributions and conclusions
  \end{itemize}

Kent Beck~\cite{JohnsonBBCGW93} proposes to have four sentences in a good abstract:

  \begin{enumerate}
    \item The first states the problem.
    \item The second states why the problem is a problem.
    \item The third is the startling sentence.
    \item The fourth states the implication of the startling sentence.
  \end{enumerate}
}

\begin{document}
\maketitle

%%%%%%%%%%%%%%%%%%%%%%%%%%%%%%%%%%%%%%%%%%%%%%%%%%%%%%%%%%%%%%%%%%%%%%%%%%%%%%%%
% Let the juicy stuff start.
%%%%%%%%%%%%%%%%%%%%%%%%%%%%%%%%%%%%%%%%%%%%%%%%%%%%%%%%%%%%%%%%%%%%%%%%%%%%%%%%

%%%%%%%%%%%%%%%%%%%%%%%%%%%%%%%%%%%%%%%%%%%%%%%%%%%%%%%%%%%%%%%%%%%%%%%%%%%%%%%%
\chapter{Introduction}
Software scalability.
Cloud scalability.
Why is it important
Importance of measuring in the clouds and testing optimal architectures.

\section{Initial Study}
What we found out researching.

\section{Problem Statement}
what's the best architecture that answers that best:
how to deliver data to geographically distributed users with minimal, consistent and manageable latency
how to react
scales up and down in response to demand changes (WEBSOCKETS - DISCONNECT SESSIONS? WHEN?  This provides some additional challenges when it comes to websockets, since in order to stop an instance one needs to make sure it does not serve any sessions. [MMPROJ9])
according to the models based on the set of metrics described in detail later \ref{Selected metrics}
Also, there is an inherent limitation for scaling a websocket application - a number of TCP ports (and file descriptors available to a ws/wss connections) on a server instance. Hardware limitations are different in an http-based application.
\subsection{Research Questions}
\begin{enumerate}
  \item Does architecture with geographically aware scaling case can produce better results than the baseline architecture?
  \item What is the preferred architecture decomposition for a system with given characteristics - stateless, deployed in the cloud, dynamically scaled up and down, with persistent connections, clients distributed globally and uncacheable data generated in real time?
\end{enumerate}
\subsection{Solution Outline}
How our solution works in short.
\subsection{Research Method}
Software engineering is a relatively difficult field to investigate in terms of lack of clarity on how to do it. Most of the problems are of design, rather than pure scientific, nature. West Churchman coined a specific term for these kind of modern problems - wicked (since they are resistant to resolutions) [MMPROJ1]. Eastbrook et al claim it is often difficult to identify the true underlying nature of the research problem and thus the best method to research it [MMPROJ2]. In their work, they name and compare five most classes of research methods to select from: controlled experiments, case studies, survey research, ethnographies, action research. They help to select the method by first establishing the type of research question being asked - existence question (“Does X exist”?), description and classification question (“What is X like?”), and descriptive-comparative questions(“How does X offer from Y?”). The questions of this research are of the last type. The authors suggest pinpointing up-front what will be accepted as a valid answer to the research question [MMPROJ2]. The detailed description of each of the methods helps me to settle for the controlled experiment. It’s well-suited for testing a hypothesis where manipulating independent variables has an effect of dependent ones, which is exactly the case of me research. The manipulated variable is architecture decomposition, and the measured ones are determined by scalability measurement models described below.
\subsubsection{Research Difficulty}
The experiments will be performed in a shared cloud environment, which is inherently unpredictable and changing. Designing and executing a test yielding statistically relevant results where all the necessary variables are controlled (within reasonable boundaries) will be a challenge.
The focus of the project is on scaling a websocket application. This is a relatively new technology, thus finding proper scientific coverage is not trivial.
Some of the necessary development and data collection might be in relatively low-level technology.
The geo-location with reasonable accuracy of the clients might prove difficult too.
Overall, it’s a complex project requiring knowledge of both hardware (protocol), software (architecture) and consisting of multiple disciplines.
even linking order in the compiler can change the performance!!! [one of CRIMES]
\subsection{Hypothesis}
The correct geographical decomposition of application stack can lead to vastly improved performance in comparison with baseline architecure(TODO:S????).

\section{Contributions}
Case study on AWS.
Architecture decomposition analysis.
Deliverable piece of code - metrics framework and auto-scaling scripts.

\section{Related Work}
The big paper on measuring scalability in the clouds.
custom routing between regions. A lot on page 14/21 MMPROJ.
The big paper on deploying different scalability structures.
Small mentions of the others.

\section{Outline}
Here we outline the structure of the thesis. A short paragraph on what happens in each chapter.
%%%%%%%%%%%%%%%%%%%%%%%%%%%%%%%%%%%%%%%%%%%%%%%%%%%%%%%%%%%%%%%%%%%%%%%%%%%%%%%%

%%%%%%%%%%%%%%%%%%%%%%%%%%%%%%%%%%%%%%%%%%%%%%%%%%%%%%%%%%%%%%%%%%%%%%%%%%%%%%%%
\chapter{References TO MOVE TO THESIS.BIB}

[Master's Project Plan]

[MMPROJ1] Churchman, C. West, "Wicked Problems", Management Science 14 (4) (December 1967).
[MMPROJ2] S.Easterbrook, J.Singer, M.A. Storey, D. Damian, Selecting Empirical Methods for Software Engineering Research, Java Report 3 (7) (1998) 51–56.
[MMPROJ9] N. Grozev, R. Buyya, Multi-Cloud Provisioning and Load Distribution for Three-Tier Applications, ACM Transactions on Autonomous and Adaptive Systems (TAAS), Vol.9 Iss.3, Article No. 13 (October 2014)
[MMPROJ10] T. Leighton, Improving performance of the internet, Communications of the ACM - Inspiring Women in Computing, Vol. 52, Iss. 2, p. 44-51 (February 2009)

%%%%%%%%%%%%%%%%%%%%%%%%%%%%%%%%%%%%%%%%%%%%%%%%%%%%%%%%%%%%%%%%%%%%%%%%%%%%%%%%

%%%%%%%%%%%%%%%%%%%%%%%%%%%%%%%%%%%%%%%%%%%%%%%%%%%%%%%%%%%%%%%%%%%%%%%%%%%%%%%%
\chapter{Background}

\section{Scalability of Software Systems}

Scalability seems to be a notion that everyone intuitively grasps, but has difficulties when it comes to clear explanations. In my literature study I came across a few similar definitions, which might help with that, of which three can be found below:

\begin{quote}
\cite{Williams04} Scalability is a measure of an application system’s ability to, without modification, cost-effectively provide increased throughput, reduced response time and/or support more users when hardware resources are added.
\end{quote}

\begin{quote}
\cite{WeinstockOnSystem2006} Scalability is an ability of a system to handle increased workload (without adding resources).
\end{quote}

\begin{quote}
\cite{WeinstockOnSystem2006} Scalability is an ability of a system to handle increased workload by repeatedly applying a cost-effective strategy for extending a system’s capacity.
\end{quote}

Scalability is generally desired in the software systems, yet it comes at a cost associated with the system's design. Design of scalable systems is more complex (since there are additional problems that need to be dealt with).

As the globalization and internetization progresses, more and more systems are expected to be capable of serving millions of globally distributed users. A single server instance often cannot live up to that task and thus application needs to be divided and distributed in multiple smaller chunks. This division happens on different application layers, and different parts of the system now have to communicate and synchronize with each other.

The user traffic, and with it the need for system services and resources, rarely stays constant. From this, a need to be able to scale up and down dynamically in response to traffic arises.

tradeoffs:
performance and scalability,
cost and scalability,
operability and scalability,
usability and scalability,
data consistency and scalability
\cite{WeinstockOnSystem2006}

Huge parts of the internet are shifting towards real-time. This trend is giving rise to new technologies for exchanging messages between clients and servers in the client-server architecture. Traditionally, client would send a request to a server and receive a response. This is hugely inefficient when there is a need for continuous bidirectional exchange of messages. As an improvement, new mechanisms for server-client communication were introduced: first long polling, to enhance the process and reduce overhead. Websockets are a huge next step on this path but introduce new challenges. One of them is scalability of applications which make use of this technology.

Before the rise of the websocket protocol, different techniques were used to improve communication between server to client. Among them, the following can be listed (some of them serving different purposes):
ajax - a request / response model; was an improvement since the page didn’t have to be refreshed anymore to get new data,
short polling - using a timer to regulate sending requests to server, similar to refreshing the page, useful when data doesn’t change too often,
long polling - now considered to be a workaround of preventing creating connections for each request, keeps a connection artificially alive for some time, clients have to reconnect periodically,
webRTC - a peer to peer connection
server-side events - only allow for a server-initiated communication.

There exist multiple scalability vectors - applications can be scaled in a multitude of different ways. Different decompositions of application stack can be applied. Database layer itself can be scaled up (or out) in numerous ways. They and their benefits are described below.

The two researchers present us with an interesting thought - software engineers need to design for the cloud, not only to deploy in it. To facilitate that process, they propose an adaptive dynamic provisioning and autonomous workload redirection algorithms. [MMPROJ9] 


\subsection{Existing Approaches}
A few traditional approaches of tackling an issue like that exist already. Among them we can enumerate:
Scaling up (one massive instance),
end-user experience is at the mercy of the unreliable Internet and its middle-mile bottlenecks
Traffic levels fluctuate tremendously, so the need to provision for peak traffic levels means that expensive infrastructure will sit underutilized most of the time
model does not provide the flexibility to handle unexpected surges [All 28]
This will be selected as the baseline architecture - first with isolating the network effects (tested on an internal network), and then in the real-life cloud scenario. This way the real impact on the performance of the architecture of network variables can be established.
Scaling out randomly - firing up new system nodes in new data centres,
Averaging out geographical locations and firing up instances 'in the middle',
Content Delivery Networks - these only handle static assets. Websockets do not support CDNs, because the WebSocket protocol is stateful.
Big Data Center CDNs
potential improvements are limited because the CDN’s servers are still far away from most users and still deliver content from the wrong side of the middle-mile bottlenecks
Highly Distributed CDNs
putting servers within end-user ISPs
Algorithmically predicting message frequency and opening / closing persistent connections according to this.
Peer to peer networks.
[MMPROJ10]
Scaling up (using a bigger server instance) is easy to implement, but costly, even extremely when you start pushing at current hardware limits. Which, with a websocket-based application is not an impossible option. Instead, one can perform horizontal scaling out and cost of hardware can be reduced dramatically this way. He discusses different types of balancing: application layer balancing, business load balancing, and anticipating load. He claims that the overhead of parsing requests in the application layer is high thus limiting scalability compared to load balancing in the transport layer. [MMPROJ14]


\section{Cloud Scalability}
General.

\section{Data-layer Scalability}

\section{Websocket Scalability}

\section{Geographical Distribution}

\section{Measuring Scalability} \label{Measuring Scalability}
What we have from literature study. ONLY THE GENERAL STUFF - WE HAVE A FULL CHAPTER \ref{Scalability Measurements} ON THIS.

\section{Benchmarking}

\subsection{Benchmarking Crimes}

%%%%%%%%%%%%%%%%%%%%%%%%%%%%%%%%%%%%%%%%%%%%%%%%%%%%%%%%%%%%%%%%%%%%%%%%%%%%%%%%

%%%%%%%%%%%%%%%%%%%%%%%%%%%%%%%%%%%%%%%%%%%%%%%%%%%%%%%%%%%%%%%%%%%%%%%%%%%%%%%%
\chapter{The System Under Test}

real time prediction parmet
parimutuel pools
directed contracts
sort of a stock market for real-time prediction on anything, e.g. live sport

\section{General Purpose of The System}
first applicatoin - live sports
low manageable latencies important

\subsection{Sample Use Case}
A good example e.g. can be predicting the outcomes of certain drives in the football match as they happen. A sample case: Manchester United - Liverpool live game, live data coming from UK servers, introduced into the system through a UK node, most users (who also generate data that needs to be distributed) in China (10k), India(10k), Australia(3k). Where should websocket servers which distribute messages spun up for minimal latencies for all clients? Is it faster to spin up db replicas on that nodes too?

\subsection{Users of the System}
in stadium
on couch
dekstop
tablet
mobile
Users of the platform are mostly on mobile networks (that often drop), so reconnecting them quick to the right (providing lowest latencies) instance is important.
lumpy demand - it comes in 2-hour-long spikes and then can go quiet for days

\section{Basic Architecture}

\section{Technology Stack}
Python + redis + javascript + mysql (Ampersand)
cloud stack described here: \section{Cloud architecture setup}


\section{Unique Aspects of The System}
critical data flows over websockets
Users receiving data shared locally should receive it at the same time as data shared globally (with as low a latency as possible). This creates a need for globally consistent and manageable latency between end user and the system.
Connected users and sources of data are geographically changing.
Users geographical center of mass is changing for each peak of demand.
Extremely lumpy demand (peaks lasting around 2 hours). This creates a need for being able to quickly scale up and scale down.
New data is generated every few seconds by the users so caching the content and distributing geographically is difficult.


%%%%%%%%%%%%%%%%%%%%%%%%%%%%%%%%%%%%%%%%%%%%%%%%%%%%%%%%%%%%%%%%%%%%%%%%%%%%%%%%

%%%%%%%%%%%%%%%%%%%%%%%%%%%%%%%%%%%%%%%%%%%%%%%%%%%%%%%%%%%%%%%%%%%%%%%%%%%%%%%%
\chapter{Experiment Outline}

\begin{enumerate}
  \item Baseline architecture on a local network.
  \item Baseline architecture deployed in the cloud.
  \item Improved architecture in the cloud.
\end{enumerate}

All of them measured with the same set of metrics with a prepared framework for gathering them.

Details below.

\section{Goal of the experiment}

The goal of the project is to design a scalability framework for a real-time persistent websocket distributed application (the system). The core researched topic will be whether a systems awareness of clients geographical distribution can improve the system performance according to selected metrics, in comparison with traditional approaches.

The goal of the project is to see if the proposed architecture decomposition can perform better (quantitatively, according to the selected metric model) than the baseline architecture in serving geographically dispersed clients. Approaches used to scale simple http applications cannot always be translated to websocket applications since the communication protocol differs. Websockets put a different kind of strain on the server machines since these need to keep the connection opened on a port for a prolonged period of time rather than simply open, server and close (as is the case with http). Along the way, an answer needs to be found what level of decoupling provides best performance on each layer of a stateless persistent system. One of the properties of a system of that kind is that key value stores come under heavy load since this is where the state resides. A good solution for distributing (sharding / replicating) these also needs to be found. Same goes for persistent storage.
TODO: REFERENCE CURRENT APPROACHES HERE.

\section{Load Testing Framework}
tsung and all tools

\section{Baseline architecture on a local network}
Architecture diagrams.
Technical details - local server capabilities.
Load testing.

\section{Cloud architecture setup}
route53
load balancers with AutoScaleGroups. vs what?
instances behind each groups. automatically scalable, connected to external redis and mysql instances.

\subsection{Baseline architecture deployed in the cloud}

\subsection{Improved architecture deployed in the cloud}

\section{Experiment deliverables}

%%%%%%%%%%%%%%%%%%%%%%%%%%%%%%%%%%%%%%%%%%%%%%%%%%%%%%%%%%%%%%%%%%%%%%%%%%%%%%%%

%%%%%%%%%%%%%%%%%%%%%%%%%%%%%%%%%%%%%%%%%%%%%%%%%%%%%%%%%%%%%%%%%%%%%%%%%%%%%%%%
\chapter{Scalability Measurements} \label{Scalability Measurements}
The previous description \ref{Measuring Scalability} concerned general stuff.
Here we describe the metrics we settled for.

Most of what is necessary for the purpose of this research has been covered by Pushkala Pattabiraman et al [MMPROJ3].
They realized that cloud computing and its measurement provides a new set of challenges when it comes to measuring performance testing, as opposed to measuring performance of  traditional software systems. They list some key points for measuring cloud applications, among them we can find: validating and ensuring the elasticity of scalability and evaluating utility service billings and pricing models. The latter is also important in my case since cost is one of the driving factors in assessing the scalability of my system. A question they raise regarding this is: How to use a transparent approach to monitor and evaluate the correctness of the utility bill based on a posted price model during system performance evaluation and scalability measurement?.
The authors divide the performance indicators into three groups:
computing resource (CPU, disk, memory, networks) - they can be helpful in establishing baseline architecture in my case,
workload indicators (connected users, throughput and latency),
performance indicators - processing speed, system reliability and scalability based on the given QoS standards.
For each they propose formal models with pluggable values and graphic representations (BELOW).
On top of that, the research contains a case study performed in the Amazon EC2 environment [MMPROJ3].
Cloud limitations need to be taken into account. One needs to be aware of hidden costs (e.g. autoscaling service is free on EC2, but it requires cloudwatch, which is not). The authors also advise to pay attention to inconsistencies in performance and scalability data [MMPROJ3].

many others:
There is much more work related to the general scalability of distributed systems. Srinivas and Janakiram in their work [MMPROJ5] mention a metric evaluating scalability as a product of throughput and response time (or any value function) divided by the cost factor. They propose another model considering scalability as a function of synchronization, consistency, availability, workload and faultload. It aims on identifying bottlenecks and hence improving the scalability. The authors also emphasize the fact of interconnectedness of synchronization, consistency and availability.
Jogalekar and Woodside [6] propose a strategy-based scalability metric based on cost effectiveness (a function of system's throughput and it’s quality of service). It separates evaluation of throughput or quantity of work from QoS (which, according to the authors, can be any suitable expression).[MMPROJ6]

PASA[MMPROJ17]

\section{Selected metrics}
latency (messaging + handhsaking)
sustainable concurrent websocket connectoins
infrastructure cost (per unit of time, supporting the same number of users)
architecture reaction speed to changes in demand (scaling up and scaling down)
dropped connections
cpu usage
memory usage
network in network out
iops reads + writes
CRAM METRICS
TODO: review
\subsection{EC2 Available metrics}

\section{Selected model}
[MMPROJ3] With CRUMs, SPMs etc.

\section{Baseline, Local network}
Lab notes from 11 of May go here:

Basic metrics
\begin{itemize}
  \item cpu, memory, disk usage (pidstat / CloudWatch)
  \item network i/o (wireshark, list others analyzed)
  \item latency, throughput, concurrent connections, messages dropped?
\end{itemize}

\subsection{Load Testing}
all the analyzed tools, list also in lab notes from may 11
\subsubsection{Users Distribution}
The authors suggest choosing randomly when generating load as to which operation to perform, on what data size etc. They suggest using different random distributions: uniform, zipfian, latest, multinomial[MMPROJ7]. Another distribution is suggested by Grozev and Buyya [MMPROJ9] - Poisson distribution with a constant mean.

\section{Baseline, Deployed in the cloud}

\section{Improved, Deployed in the cloud}
%%%%%%%%%%%%%%%%%%%%%%%%%%%%%%%%%%%%%%%%%%%%%%%%%%%%%%%%%%%%%%%%%%%%%%%%%%%%%%%%

%%%%%%%%%%%%%%%%%%%%%%%%%%%%%%%%%%%%%%%%%%%%%%%%%%%%%%%%%%%%%%%%%%%%%%%%%%%%%%%%
\chapter{Experiment results}
%%%%%%%%%%%%%%%%%%%%%%%%%%%%%%%%%%%%%%%%%%%%%%%%%%%%%%%%%%%%%%%%%%%%%%%%%%%%%%%%

%%%%%%%%%%%%%%%%%%%%%%%%%%%%%%%%%%%%%%%%%%%%%%%%%%%%%%%%%%%%%%%%%%%%%%%%%%%%%%%%
\chapter{Evaluation}
%%%%%%%%%%%%%%%%%%%%%%%%%%%%%%%%%%%%%%%%%%%%%%%%%%%%%%%%%%%%%%%%%%%%%%%%%%%%%%%%

%%%%%%%%%%%%%%%%%%%%%%%%%%%%%%%%%%%%%%%%%%%%%%%%%%%%%%%%%%%%%%%%%%%%%%%%%%%%%%%%
\chapter{Conclusion}
%%%%%%%%%%%%%%%%%%%%%%%%%%%%%%%%%%%%%%%%%%%%%%%%%%%%%%%%%%%%%%%%%%%%%%%%%%%%%%%%

%%%%%%%%%%%%%%%%%%%%%%%%%%%%%%%%%%%%%%%%%%%%%%%%%%%%%%%%%%%%%%%%%%%%%%%%%%%%%%%%
\chapter{Further work}
%%%%%%%%%%%%%%%%%%%%%%%%%%%%%%%%%%%%%%%%%%%%%%%%%%%%%%%%%%%%%%%%%%%%%%%%%%%%%%%%

%%%%%%%%%%%%%%%%%%%%%%%%%%%%%%%%%%%%%%%%%%%%%%%%%%%%%%%%%%%%%%%%%%%%%%%%%%%%%%%%
% BELOW WE HAVE ORIGINAL TEMPLATE STUFF THAT NEEDS TO BE REMOVED
%%%%%%%%%%%%%%%%%%%%%%%%%%%%%%%%%%%%%%%%%%%%%%%%%%%%%%%%%%%%%%%%%%%%%%%%%%%%%%%%
\chapter{Everything below that needs to be removed (except for bib)}
\chapter{Front Matter}

The first thing is to connect the class by saying:

\begin{snippet}
\begin{verbatim}
\documentclass{uvamscse}
\end{verbatim}
\end{snippet}

\section{Title}

Specify the title of the thesis with \cmd{title} and \cmd{subtitle} commands:

\begin{snippet}
\begin{verbatim}
\title{MetaThesis}
\subtitle{A Thesis Template Leading by Example}
\end{verbatim}
\end{snippet}

Any thesis can survive without a \cmd{subtitle}, but the \cmd{title} is mandatory.

\section{Author}

Introduce yourself with \cmd{author} and \cmd{authemail}:

\begin{snippet}
\begin{verbatim}
\author{Vadim Zaytsev}
\authemail{vadim@grammarware.net}
\end{verbatim}
\end{snippet}

Again, \cmd{authemail} is not mandatory. If you need anything fancier, just put it inside \cmd{author}.

\begin{snippet}
\begin{verbatim}
\author{Vadim Zaytsev\footnote{Yes, that one.}}
\end{verbatim}
\end{snippet}

The footnote would be printed on the bottom of the title page, and will be
referred to by a symbol, not by a number as any footnotes within the main
document body.

\section{Date}

By default, the date inserted in your PDF is the day of the build, e.g., ``March 25, 2014''. If you want it to be formatted differently or be more vague or outright fake, use \cmd{date}:

\begin{snippet}
\begin{verbatim}
\date{Spring 2014}
\end{verbatim}
\end{snippet}

The argument is just a string, the format is unrestricted:

\begin{snippet}
\begin{verbatim}
\date{Tomorrow. Honestly.}
\end{verbatim}
\end{snippet}

\section{Host}

If your hosting organisation is not the UvA, specify it with \cmd{host}. The
logo on the bottom of the title page will still be the UvA one, because this
is the organisation guaranteeing your degree.

\begin{snippet}
\begin{verbatim}
\host{Grammarware, Inc., \url{http://grammarware.github.io}}
\end{verbatim}
\end{snippet}

NB: footnotes will not work, unless you know how to \cmd{protect} them.

\section{Cover picture}

If the first page of your thesis looks too blunt, add a picture to it:

\begin{snippet}
\begin{verbatim}
\coverpic{figures/terminal.png}
\end{verbatim}
\end{snippet}

You can even specify the picture's width as an optional argument:

\begin{snippet}
\begin{verbatim}
\coverpic[100pt]{figures/terminal.png}
\end{verbatim}
\end{snippet}

How these three options look, you can see from \autoref{fig:titles}.

\begin{figure}[t]
  \fbox{\includegraphics[width=.25\textwidth]{figures/title1.pdf}}
  \hfill
  \fbox{\includegraphics[width=.25\textwidth]{figures/title2.pdf}}
  \hfill
  \fbox{\includegraphics[width=.25\textwidth]{figures/title3.pdf}}
  \caption{A hypothetical thesis title page without a cover picture (on the left), with an overly large one (in the centre) and with a tiny pic (on the right).}
  \label{fig:titles}
\end{figure}

\section{Abstract}

A thesis is fine without an abstract, if you do not feel like writing it and
your supervisor does not feel like enforcing it. If you do want an abstract,
make it with the \cmd{abstract} command:

\begin{snippet}
\begin{verbatim}
\abstract{This is not a thesis.}
\end{verbatim}
\end{snippet}

The abstract is just like any other section of your thesis, so you can use any
\LaTeX\ tricks there. If you think that the name ``abstract'' is too abstract
for your abstract, you can still use \cmd{abstract} without being too
abstract:

\begin{snippet}
\begin{verbatim}
\abstract[Confession]{I am a cenosillicaphobiac.}
\end{verbatim}
\end{snippet}

Kent Beck~\cite{JohnsonBBCGW93} proposes to have four sentences in a good abstract:

\begin{enumerate}
  \item The first states the problem.
  \item The second states why the problem is a problem.
  \item The third is the startling sentence.
  \item The fourth states the implication of the startling sentence.
\end{enumerate}

In practice, each of these ``sentences'' can be longer than an actual
sentence, but it is in general a good rule of thumb to condense the summary of
your thesis into these four tiny messages. Do not write too much, make it
tweetable.

%%%%%%%%%%%%%%%%%%%%%%%%%%%%%%%%%%%%%%%%%%%%%%%%%%%%%%%%%%%%%%%%%%%%%%%%%%%%%%%
\chapter{Core Chapters}

The structure of your thesis is up to you and your supervisor. Whatever you
do, do not consider the guidelines below as dogmas.

\section{Classic structure}

\begin{description}
  \item[Problem statement and motivation.]
  You describe in detail what problem the research is addressing, and what is
the motivation to address this problem. There is a concise and objective
statement of the research questions, hypotheses and goals. It is made clear
why these questions and goals are important and relevant to the world outside
the university (assuming it exists). You can already split the main research
question into subquestions in this chapter. This section also describes an
analysis of the problem: where does it occur and how, how often, and what are
the consequences? An important part is also to scope the research: what
aspects are included and what aspects are deliberately left out, and why?
  \item[Research method.]
  Here you describe the methods used to answer the research questions. A good
structure of this section often follows the subquestions by providing a method
for each. The research method needs a thorough motivation grounded in theory
in order to be acceptable. As a part of the method, you can introduce a number
of hypotheses --- these will be tested by the research, using the methods
described here. An important part of this section is validation. How will you
evaluate and validate the outcomes of the research?
  \item[Background and context.]
  This chapter contains all the information needed to put the thesis into
context. It is common to use a revised version of your literature survey for
this purpose. It is important to explicitly refer from your text to sources
you have used, they will be listed in your bibliography. For example, you can
write ``A small number of programming languages account for most language
use~\cite{MeyerovichR2013}'', where the following entry would be included in
your bibliography:
\begin{quote}
\cite{MeyerovichR2013} Leo A. Meyerovich and Ariel S. Rabkin. Empirical Analysis of Programming Language Adoption. In \emph{Proceedings of the 2013 ACM SIGPLAN International Conference on Object Oriented Programming Systems Languages and Applications}, OOPSLA, pages 1--18. ACM, 2013. \doi{10.1145/2509136.2509515}.
\end{quote}
Have a look at \autoref{sec:biblio} to learn more about citation.
  \item[Research.]
  This chapter reports on the execution of the research method as described in
an earlier chapter. If the research has been divided into phases, they are
introduced, reported on and concluded individually. If needed, this chapter
could be split up to balance out the sizes of all chapters.
  \item[Results.]
  This chapter presents and clarifies the results obtained during the
  research. The focus should be on the factual results, not the interpretation
  or discussion. Tables and graphics should be used to increase the clarity of
  the results where applicable.
  \item[Analysis and conclusions.]
  This chapter contains the analysis and interpretation of the results. The
  research questions are answered as best as possible with the results that
  were obtained. The analysis also discussed parts of the questions that were
  left unanswered.

  An important topic is the validity of the results. What methods of
  validation were used? Could the results be generalised to other cases? What
  threats to validity can be identified? There is room here to discuss the
  results of related scientific literature here as well. How do the results
  obtained here relate to other work, and what consequences are there? Did
  your approach work better or worse? Did you learn anything new compared to
  the already existing body of knowledge? Finally, what could you say in
  hindsight on the research approach by followed? What could have done better?
  What lessons have been learned? What could other researchers use from your
  experience? A separate section should be devoted to ``future work'', i.e.,
  possible extension points of your work that you have identified. Even other
  researchers should be able to use those as a starting point.
\end{description}

\section{Reporting on replications}

Here are the guidelines to report on replicated studies~\cite{Carver10}:

\begin{description}
  \item[Information about the original study]~\\
    \begin{description}
    \item[Research question(s)] that were the basis for the design
    \item[Participants,] their number and any other relevant characteristics
    \item[Design] as a graphical or textual description of the experimental design
    \item[Artefacts,] the description of them and/or links to the artefacts used
    \item[Context variables] as any important details that affected the design of the study or interpretation of the
results
    \item[Summary of the results] in a brief overview of the major findings
    \end{description}
  %
  \item[Information about the replication]~\\
    \begin{description}
    \item[Motivation for conducting the replication] as a
description of why the replication was conducted:
to validate the results, to broaden the results by
changing the participant pool or the artifacts.
    \item[Level of interaction with original experimenters.]
The level of interaction between the original experimenters and the
replicators should be reported. This interaction could range from none (i.e.
simply read the  paper) to them being the same people. There is quite a lot of
discussion of the level of interaction allowed for the replication to be
``successful'', but this level should be reported even without  addressing
the controversy.
    \item[Changes to the original experiment.] Any changes made to the
design, participants, artifacts, procedures, data collected and/or analysis
techniques should be  discussed along with the motivation for the change.
    \end{description}
  \item[Comparison of results to original]~\\
    \begin{description}
    \item[Consistent results,] when replication results supported
results from the original study, and
    \item[Differences in results,] when results from the replication
did not coincide with the results from the original study.
Authors should also discuss how changes made to the
experimental design (see above) may have caused
these differences.
    \end{description}
    \item[Drawing conclusions across studies]
\end{description}

NB: this section contains portions of text repeated directly from Carver~\cite{Carver10} and
only slightly massaged. Do not do this for your thesis, write your own thoughts down.

\section{\LaTeX\ details}

\subsection{Environments}

A \LaTeX\ environment is something with opening and closing tags, which look
like \cmd{begin}\{\texttt{name}\} and \cmd{end}\{\texttt{name}\}. Some useful
environments to know:

\begin{center}
\begin{tabular}{ll}
  \texttt{itemize}      & bullet lists\\
  \texttt{enumerate}    & numbered lists\\
  \texttt{description}  & definition lists\\
  \hline
  \texttt{center}       & centered line elements\\
  \texttt{flushright}   & right aligned lines\\
  \texttt{flushleft}    & left aligned lines\\
  \hline
  \texttt{tabular}      & table\\
  \texttt{longtable}    & multi-page table (needs the \texttt{longtable} package)\\
  \texttt{sideways}     & rotates some text\\
  \texttt{quote}        & block quote\\
  \texttt{verbatim}     & unformatted text\\
  \texttt{minipage}     & compound box with elements inside\\
  \texttt{boxedminipage}& compound box with elements inside and a border around it\\
  \hline
  \texttt{table}        & floating table (needs to have \texttt{tabular} nested inside)\\
  \texttt{figure}       & floating figure\\
  \texttt{sourcecode}   & floating listing\\
  \hline
  \texttt{equation}     & mathematical equation\\
  \texttt{lstlisting}   & pretty-printed syntax highligted listing\\
  \texttt{multline}     & mathematical equation spanning over multiple lines\\
  \texttt{eqnarray}     & system of mathematical equations\\
  \texttt{gather}       & bundled mathematical equations\\
  \texttt{align}        & bundled and aligned mathematical equations\\
  \texttt{array}        & matrix\\
  \texttt{CD}           & commutative diagrams\\
\end{tabular}
\end{center}

\section{Listings}

\begin{sourcecode}
\begin{lstlisting}[language=prolog]
define(Ps1,G1,G2)
 :-
    usedNs(G1,Uses),
    ps2n(Ps1,N),
    require(
      member(N,Uses),
      'Nonterminal ~q must not be fresh.',
      [N]),
    new(Ps1,N,G1,G2),
    !.
\end{lstlisting}
\caption{Code in Prolog}
\end{sourcecode}

\begin{sourcecode}
\begin{lstlisting}[language=sdf]
module Syntax

imports Numbers
imports basic/Whitespace

exports
  sorts
    Program Function Expr Ops Name Newline

  context-free syntax
    Function+                          -> Program
    Name Name+ "=" Expr Newline+       -> Function
    Expr Ops Expr                      -> Expr      {left,prefer,cons(binary)}
    Name Expr+                         -> Expr      {avoid,cons(apply)}
    "if" Expr "then" Expr "else" Expr  -> Expr      {cons(ifThenElse)}
    "(" Expr ")"                       -> Expr      {bracket}
    Name                               -> Expr      {cons(argument)}
    Int                                -> Expr      {cons(literal)}
    "-"                                -> Ops       {cons(minus)}
    "+"                                -> Ops       {cons(plus)}
    "=="                               -> Ops       {cons(equal)}
\end{lstlisting}
\caption{Code in SDF}
\end{sourcecode}

\begin{sourcecode}
\begin{lstlisting}[language=Java]
import types.*;
import org.antlr.runtime.*;

public class TestEvaluator
    public static void main(String[] args) throws Exception {

        // Parse file to program
        ANTLRFileStream input = new ANTLRFileStream(args[0]);
        FLLexer lexer = new FLLexer(input);
        CommonTokenStream tokens = new CommonTokenStream(lexer);
        FLParser parser = new FLParser(tokens);
        Program program = parser.program();

        // Parse sample expression
        input = new ANTLRFileStream(args[1]);
        lexer = new FLLexer(input);
        tokens = new CommonTokenStream(lexer);
        parser = new FLParser(tokens);
        Expr expr = parser.expr();

        // Evaluate program
        Evaluator eval = new Evaluator(program);
        int expected = Integer.parseInt(args[2]);
\end{lstlisting}
\caption{Code in Java}
\end{sourcecode}

\begin{sourcecode}
\begin{lstlisting}[style=mono,language=Python]
#!/usr/local/bin/python
# wiki: BGF
import os
import sys
import slpsns
import elementtree.ElementTree as ET

# root::nonterminal* production*
class Grammar:
  def __init__(self):
    self.roots = []
    self.prods = []
  def parse(self,fname):
    self.roots = []
    self.prods = []
    self.xml = ET.parse(fname)
    for e in self.xml.findall('root'):
      self.roots.append(e.text)
    for e in self.xml.findall(slpsns.bgf_('production')):
      prod = Production()
      prod.parse(e)
      self.prods.append(prod)
\end{lstlisting}
\caption{Code in Python}
\end{sourcecode}

\chapter{Literature}\label{sec:biblio}

\textsc{Bib}TeX\ is a JSON-like format for bibliographic entries. Encode each
source once as a \textsc{Bib}\TeX\ entry, give it a name and refer to it from
any place in your thesis. The bibliography at the end of the thesis will be
compiled automatically from those entries that are referenced at least once,
it will also be automatically sorted and fancyfied (URLs, DOIs, etc).

DOI is a digital object identifier, it is uniquely and immutably assigned to
any paper published in a well-established journal or conference proceedings
and can be used to refer to it. When used in a browser, it resolves to a
publisher's website where paper can be obtained. Including DOIs in citations
is considered good practice and lets the readers of your thesis get to the
text of the paper in one click. Books do not have DOIs, only ISBNs; some
workshop proceedings and most unofficial publications do not have DOIs. If you
want to get a DOI assigned to your work such as a piece of code, upload it to
\href{http://www.figshare.com}{FigShare}.

Keys in key-value pairs within each \textsc{Bib}\TeX\ entry are never quoted,
values usually are, but can also be included within curly brackets or left as
is, which works fine for numbers (e.g., years). If you want to preserve the
value from any adjustments (e.g., no recapitalisation in titles), use curlies
\emph{and} quotes. Separate authors and editors by ``and'', which will
automatically be mapped to commas or left as ``and''s as necessary.

\section{Books}

\cite{GruneJacobs} is just as good as the Dragon Book, but newer and has an
awesome extended bibliography available for free.

\begin{snippet}
\begin{verbatim}
@book{GruneJacobs,
  author    = "D. Grune and C. J. H. Jacobs",
  title     = "{Parsing Techniques: A Practical Guide}",
  series    = "Monographs in Computer Science",
  edition   = 2,
  publisher = "Springer",
  url       = "http://www.cs.vu.nl/~dick/PT2Ed.html",
  year      = 2008,
}
\end{verbatim}
\end{snippet}

\section{Journal papers}

Not all TOSEM papers are hard to read~\cite{GrammarwareAgenda}.

\begin{snippet}
\begin{verbatim}
@article{GrammarwareAgenda,
  author      = "Paul Klint and Ralf L{\"a}mmel and Chris Verhoef",
  title       = "{Toward an Engineering Discipline for Grammarware}",
  journal     = "ACM Transactions on Software Engineering Methodology (TOSEM)",
  volume      = 14,
  number      = 3,
  year        = 2005,
  pages       = "331--380",
}
\end{verbatim}
\end{snippet}

\section{Conference papers}

There is no limit to how many grammars can be used in one paper, but the
current record stands at 569~\cite{Micropatterns2013}.

\begin{snippet}
\begin{verbatim}
@inproceedings{Micropatterns2013,
  author = "Vadim Zaytsev",
  title = "{Micropatterns in Grammars}",
  booktitle = "{Proceedings of the Sixth International Conference on Software Language Engineering
                (SLE 2013)}",
  year = 2013,
  editor = "Martin Erwig and Richard F. Paige and Eric Van Wyk",
  volume = "8225",
  series = "LNCS",
  pages = "117--136",
  address = "Switzerland",
  month = oct,
  publisher = "Springer International Publishing",
  doi = "10.1007/978-3-319-02654-1_7",
}
\end{verbatim}
\end{snippet}

\section{Theses}

The seventh PhD student of Paul Klint was Jan Rekers~\cite{Rekers92}.

\begin{snippet}
\begin{verbatim}
@phdthesis{Rekers92,
 author   = "J. Rekers",
 title    = "{Parser Generation for Interactive Environments}",
 school   = "University of Amsterdam",
 year     = 1992,
 url      = "http://homepages.cwi.nl/~paulk/dissertations/Rekers.pdf",
}
\end{verbatim}
\end{snippet}

There is also \texttt{mastersthesis} type with exactly the same structure for
referring to Master's theses.

\section{Technical reports}

The original seminal work introducing two-level grammars was never published
in any book or conference, but there is a technical report explaining
it~\cite{Wijngaarden65}. SMC, or \emph{Stichting Matematisch Centrum}, was the
old name of CWI fifty years ago.

\begin{snippet}
\begin{verbatim}
@techreport{Wijngaarden65,
        author      = "Adriaan van Wijngaarden",
        title       = "{Orthogonal Design and Description of a Formal Language}",
        month       = oct,
        year        = 1965,
        institution = "SMC",
        type        = "{MR 76}",
        url         = "http://www.fh-jena.de/~kleine/history/languages/VanWijngaarden-MR76.pdf",
}
\end{verbatim}
\end{snippet}

\section{Wikipedia}

You do not refer to Wikipedia from academic writing, it works the other way around.

\section{Anything else}

You can refer to pretty much anything (websites, blog posts, software) through
\texttt{misc} type of entry~\cite{ANTLR}:

\begin{snippet}
\begin{verbatim}
@misc{ANTLR,
 author       = "Terence Parr",
 title        = "{ANTLR---ANother Tool for Language Recognition}",
 howpublished = "Software",
 url          = "http://antlr.org",
 year         = "2008"
}
\end{verbatim}
\end{snippet}

{%\tiny
\bibliographystyle{alphaurl}
\bibliography{thesis}
}

\end{document}
